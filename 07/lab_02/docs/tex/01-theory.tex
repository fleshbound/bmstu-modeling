\section{Теоретическая часть}

\subsection{Марковский процесс}

Для математического описания функционирования устройств, развивающихся в форме случайного процесса, может быть применен математический аппарат, разработанный в теории вероятностей для марковских случайных процессов.
Случайный процесс, протекающий в некоторой системе, называется \textit{марковским}, если для каждого момента времени вероятность любого состояния системы в будущем зависит только от состояния системы в настоящем и не зависит от того, когда и каким образом система пришла в это состояние.
В реальности таких систем не существует.

В марковском случайном процессе будущее развитие зависит только от настоящего состояния и не зависит от предыстории процесса.
Для марковского случайного процесса составляют уравнения Колмогорова, представляющие собой соотношения следующего вида:
\begin{equation}
	F(P'(t), P(t), \lambda) = 0,
\end{equation}
где $P(t)$ --- вероятность нахождения в состоянии для сложной системы,
$\lambda$ --- коэффициенты, показывающие, с какой скоростью система переходит из одного состояния в другое (интенсивность).