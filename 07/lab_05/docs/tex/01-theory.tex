\section{Условие лабораторной}

Моделируем информационный центр. В информационный центр приходят клиенты (пользователи) через интервал времени 10 ± 2 минуты. Если все три имеющихся оператора заняты, клиенту отказывают в обслуживании.
Операторы имеют разную производительность и могут обеспечивать обслуживание среднего запроса от пользователя за 20 ± 5, 40 ± 10 и 40 ± 20 ед. времени (минут).
Клиенты стараются занять свободного оператора с максимальной производительностью.
Полученные запросы сдаются в накопитель, откуда выбираются на обработку.
На первый компьютер — от первого и второго операторов, на второй --- от третьего.
Время обработки запроса в компьютерах — 15 и 30 минут соответственно.
Смоделировать процесс обработки 300 запросов. Определить вероятность отказа.

\section{Теоретическая часть}

\subsection{Схемы модели}

На рисунке \ref{img:blockDiagram} представлена структурная схема модели.

\begin{figure}[ht!]
	\centering
	\includegraphics[width=0.6\linewidth]{../img/blockDiagram.png}
	\caption{Структурная схема модели}
	\label{img:blockDiagram}
\end{figure}

В процессе взаимодействия клиентов с информационным центром возможно два режима работы:

\begin{itemize}
	\item режим нормального обслуживания, когда клиент выбирает одного из свободных операторов, отдавая предпочтение тому, у кого максимальная производительность;
	\item режим отказа клиенту в обслуживании, когда все операторы заняты.
\end{itemize}

На рисунке \ref{img:queuingSystems} представлена схема модели в терминах систем массового обслуживания (СМО).

\begin{figure}[ht!]
	\centering
	\includegraphics[width=0.6\linewidth]{../img/queuingSystems.png}
	\caption{Схема модели в терминах СМО}
	\label{img:queuingSystems}
\end{figure}

\subsection {Равномерное распределение}

Случайная величина $X$ имеет \textit{равномерное распределение} на отрезке~$[a,~b]$, если ее плотность распределения~$f(x)$ равна:
\begin{equation}
	p(x) =
	\begin{cases}
		\displaystyle\frac{1}{b - a}, & \quad \text{если } a \leq x \leq b;\\
		0,  & \quad \text{иначе}.
	\end{cases}
\end{equation}

При этом функция распределения~$F(x)$ равна:

\begin{equation}
	F(x) =
	\begin{cases}
		0,  & \quad x < a;\\
		\displaystyle\frac{x - a}{b - a}, & \quad a \leq x \leq b;\\
		1,  & \quad x > b.
	\end{cases}
\end{equation}

Обозначение: $X \sim R[a, b]$.

\begin{equation}
	T_{i} = a + (b - a) \cdot R,
\end{equation}

\noindent где $R$ --- псевдослучайное число от 0 до 1.

\subsection{Переменные и уравнение имитационной модели}

\textbf{Эндогенные переменные:}

\begin{itemize}
	\item время обработки задания $i$-ым оператором;
	\item время решения задания на $j$-ом компьютере.
\end{itemize}

\textbf{Экзогенные переменные:}

\begin{itemize}
	\item $n0$ — число обслуженных клиентов;
	\item $n1$ — число клиентов, получивших отказ.
\end{itemize}

Вероятность отказа в обслуживании клиента будет вычисляться как:

\begin{equation}
	P = \frac{n_0}{n_0 + n_1}
\end{equation}

