\section{Аналитическая часть}

Говорят, что непрерывная случайная величина $X$ имеет \textbf{равномерное распределение на отрезке $[a, b]$}, если ее функция плотности распределения имеет вид (\ref{eq:cont}):

\begin{equation}\label{eq:cont}
	f(x) = \begin{cases}
			\cfrac{1}{b - a}, & x \in [a, b]; \\
			0, & \text{иначе.}
	\end{cases}
\end{equation}

Функция распределения при этом равна (\ref{eq:contF}):

\begin{equation}\label{eq:contF}
	F(x) = \begin{cases}
		0, & x < a; \\
		\cfrac{x - a}{b - a}, & x \in [a, b]; \\
		1, & x > b.
	\end{cases}
\end{equation}

Обозначение: $X \sim R[a, b]$.

Равномерное распределение используется при рассмотрении равновероятных событий.
Примеры: время ожидания транспорта, ошибки округления в пределах цены деления.

Говорят, что случайная величина $X$ имеет \textbf{распределение Пуассона с параметром $\lambda > 0$}, если $X$ принимает значения $0, 1, 2, \dots, e$ с вероятностями 

\begin{equation}
	P\{X=k\} = \frac{\lambda^k}{k!} e^{-\lambda}, k \in \mathbb{R}_0.
\end{equation}

Функция распределения:

\begin{equation}
	F(x) = \frac{\lambda^x}{x!} e^{-\lambda}
\end{equation}

Функция плотности распределения: 

\begin{equation}
	f(x) = \sum\limits_{i = 0}^{x} \frac{\lambda^i}{i!} e^{-\lambda}
\end{equation}

Обозначение: $X \sim \Pi(\lambda)$.

Пуассоновское распределение используется при рассмотрении потока событий, наступающих независимо друг от друга с фиксированной средней интенсивностью $\lambda$.
Примеры: поток сетевых пакетов, попытки входа в систему.
